\documentclass[a4paper,12pt]{article}
\usepackage{amsmath}
\usepackage{amssymb}
\usepackage{graphicx}
\usepackage{geometry}
\geometry{a4paper, margin=1in}

\title{Berechnung des Wärmebedarfs von Gebäuden basierend auf CSV-Daten}
\author{Dipl.-Ing. (FH) Jonas Pfeiffer}
\date{2024-09-09}

\begin{document}

\maketitle

\section{Einleitung}
Dieses Dokument beschreibt die Methodik zur Berechnung von Heizlastprofilen für Gebäude basierend auf CSV-Daten. Die Implementierung beinhaltet Funktionen zur Ermittlung des Heizwärme- und Warmwasserbedarfs von Gebäuden, welche die Berechnungsmethoden VDI 4655 und BDEW verwenden. Zusätzlich werden die Vor- und Rücklauftemperaturen der Heizsysteme berechnet.

\section{Funktion \texttt{generate\_profiles\_from\_csv}}

Die Funktion \texttt{generate\_profiles\_from\_csv} berechnet die Heizprofile eines Gebäudes auf Grundlage der folgenden Eingabedaten:

\begin{itemize}
    \item \textbf{data}: Ein DataFrame mit Gebäudeinformationen, insbesondere:
    \begin{itemize}
        \item Wärmebedarf in kWh
        \item Gebäudetyp
        \item Subtyp
        \item Anteil des Warmwasserbedarfs am Gesamtwärmebedarf
        \item Normaußentemperatur
    \end{itemize}
    \item \textbf{TRY}: Pfad zu den Testreferenzjahresdaten (TRY), die stündliche Wetterdaten (z.B. Lufttemperaturen) enthalten.
    \item \textbf{calc\_method}: Die Berechnungsmethode zur Ermittlung des Wärmebedarfs, basierend auf dem Gebäudetyp oder einer angegebenen Methode.
\end{itemize}

\subsection{Berechnungslogik}
Die Funktion führt folgende Schritte aus:

\subsubsection{Aufteilung des Gesamtwärmebedarfs}

Der Gesamtwärmebedarf wird in Heizwärme und Warmwasserbedarf aufgeteilt:

\[
\text{Heizwärmebedarf} = \text{Gesamtwärmebedarf} \times (1 - \text{Warmwasserbedarf})
\]
\[
\text{Warmwasserbedarf} = \text{Gesamtwärmebedarf} \times \text{Warmwasserbedarf}
\]

\subsubsection{Berechnungsmethoden}
Je nach Gebäudetyp wird die Berechnung entweder nach der Methode VDI 4655 oder BDEW durchgeführt.

\paragraph{VDI 4655}
Für bestimmte Gebäudetypen, wie Einfamilienhäuser (EFH) und Mehrfamilienhäuser (MFH), kann die Methodik nach VDI 4655 verwendet werden. Diese Methode berechnet viertelstündliche Lastprofile für Heizung, Warmwasser und Strom. Eine ausführliche Beschreibung erfolgt in Kapitel \ref{sec:heat_load_VDI4655_doc}.

\paragraph{BDEW}
Die Methodik der Standardlastprofile nach BDEW bieten hingegen deutlich heterogene Gebäudetypen. Neben Ein- und Mehrfamilienhäusern (HEF, HMF) sind das Gebäudenutzungstypen wie Gewerbebauten oder Bürobauten. Die Berechnungsmethode gibt stündliche Lastprofile für Heizung und Warmwasser aus. Eine ausführliche Beschreibung erfolgt in Kapitel \ref{sec:heat_load_BDEW_doc}.

Die spezifische Berechnungsmethode wird basierend auf dem Gebäudetyp ausgewählt.

\subsubsection{Korrektur negativer Lasten}
Um physikalisch unsinnige negative Werte zu vermeiden, werden alle negativen stündlichen Lasten auf 0 gesetzt:

\[
\text{hourly\_heat\_demand\_total\_kW} = \max(0, \text{hourly\_heat\_demand\_total\_kW})
\]

\subsubsection{Umrechnung in Watt}
Die berechneten Lasten in kW werden in Watt umgerechnet:

\[
\text{total\_heat\_W} = \text{hourly\_heat\_demand\_total\_kW} \times 1000
\]

\subsection{Ausgabe}
Die Funktion gibt folgende Werte zurück:
\begin{itemize}
    \item \textbf{yearly\_time\_steps}: Stündliche Zeitpunkte über das Jahr hinweg.
    \item \textbf{total\_heat\_W}: Gesamtwärmelast in Watt.
    \item \textbf{heating\_heat\_W}: Heizwärmelast in Watt.
    \item \textbf{warmwater\_heat\_W}: Warmwasserlast in Watt.
    \item \textbf{max\_heat\_requirement\_W}: Maximaler Wärmebedarf in Watt.
    \item \textbf{supply\_temperature\_curve}: Vorlauftemperaturkurve des Gebäudes.
    \item \textbf{return\_temperature\_curve}: Rücklauftemperaturkurve des Gebäudes.
    \item \textbf{hourly\_air\_temperatures}: Stündliche Außentemperaturen.
\end{itemize}

\section{Funktion \texttt{calculate\_temperature\_curves}}

Diese Funktion berechnet die Vor- und Rücklauftemperaturkurven eines Gebäudes basierend auf den stündlichen Lufttemperaturen.

\subsection{Eingabe}
\begin{itemize}
    \item \textbf{data}: Ein DataFrame mit den Vor- und Rücklauftemperaturen der Heizsysteme sowie den Steigungen der Heizkurve für jedes Gebäude.
    \item \textbf{hourly\_air\_temperatures}: Array mit stündlichen Außentemperaturen.
\end{itemize}

\subsection{Berechnung der Temperaturkurven}

\subsubsection{Temperaturdifferenz}
Die Differenz zwischen Vor- und Rücklauftemperatur \( \Delta T \) wird für jedes Gebäude berechnet:

\[
\Delta T = VLT_{\text{max}} - RLT_{\text{max}}
\]

Dabei ist \( VLT_{\text{max}} \) die maximale Vorlauftemperatur und \( RLT_{\text{max}} \) die maximale Rücklauftemperatur des Gebäudes.

\subsubsection{Vorlauftemperaturkurve}
Die Vorlauftemperatur wird basierend auf der Außentemperatur und der Steigung der Heizkurve \( s \) berechnet. Für Außentemperaturen unterhalb der Normaußentemperatur \( T_{\text{min}} \) bleibt die Vorlauftemperatur konstant:

\[
\text{Vorlauftemperatur} = VLT_{\text{max}}, \quad \text{wenn } T_{\text{außen}} \leq T_{\text{min}}
\]

Wenn die Außentemperatur \( T_{\text{außen}} \) größer ist als \( T_{\text{min}} \), wird die Vorlauftemperatur gemäß folgender Gleichung angepasst:

\[
\text{Vorlauftemperatur} = VLT_{\text{max}} + s \times (T_{\text{außen}} - T_{\text{min}})
\]

\subsubsection{Rücklauftemperaturkurve}
Die Rücklauftemperaturkurve wird durch Subtraktion der Temperaturdifferenz \( \Delta T \) von der Vorlauftemperatur berechnet:

\[
\text{Rücklauftemperatur} = \text{Vorlauftemperatur} - \Delta T
\]

\section{Berechnungsmethode: VDI 4655}
\label{sec:vdi4655}

Die VDI-Richtlinie 4655 beschreibt Verfahren zur Berechnung von Lastprofilen für Wohngebäude, insbesondere im Hinblick auf Heizwärme-, Warmwasser- und Strombedarfe. Das Ziel des hier beschriebenen Algorithmus ist es, diese Lastprofile auf Basis von Testreferenzjahrsdaten (TRY), Gebäude- und Haushaltsinformationen sowie Temperatur- und Wolkendaten zu erstellen.

Die Methode zur Berechnung von Wärmebedarfsprofilen basiert auf einer Aufteilung des Jahresbedarfs in Viertelstundenintervalle, wobei Faktoren für typische Verbrauchstage sowie saisonale und witterungsbedingte Einflüsse berücksichtigt werden.

\subsection{Grundlage des Berechnungsalgorithmus nach VDI 4655}

Die Berechnungsmethode der VDI 4655 orientiert sich an der Aufteilung des Jahresenergieverbrauchs (Heizung, Warmwasser, Strom) in detaillierte Viertelstundenprofile. Diese Profile werden durch die Kombination von Temperatur- und Wolkendaten mit typischen Verbrauchsprofilen für Gebäude- und Haushaltsarten erstellt. Die Berücksichtigung von saisonalen Schwankungen und verschiedenen Klimazonen ermöglicht eine realitätsnahe Simulation des Energieverbrauchs.

\subsubsection{Testreferenzjahr (TRY) und Wetterdaten}
Die Basis der Berechnungen bildet das sogenannte Testreferenzjahr (TRY), das Wetterdaten wie stündliche Temperaturen und Bewölkungsgrade enthält. Diese Daten werden verwendet, um den Einfluss der Außenbedingungen auf den Heizwärmebedarf sowie auf den Strombedarf für die Warmwasserbereitung zu modellieren.

Die Wetterdaten werden aus einer TRY-Datei eingelesen, und es werden die folgenden Größen extrahiert:
\begin{itemize}
    \item \textbf{Temperatur (T)}: Die stündliche Außentemperatur wird zur Berechnung des Heizbedarfs genutzt.
    \item \textbf{Bewölkungsgrad (N)}: Der Bewölkungsgrad beeinflusst den Strombedarf für Licht und Geräte sowie den Heizbedarf.
\end{itemize}

\subsubsection{Definition von Nutzungsprofilen}
Für verschiedene Gebäudetypen und Haushaltsgrößen (z.B. Einfamilienhäuser, Mehrfamilienhäuser) werden in der VDI 4655 typische Tagesprofile definiert. Diese Profile spiegeln das Nutzungsverhalten über den Tag hinweg wider und variieren je nach Gebäudetyp, Tag (Werk-, Wochen- oder Feiertag) und Jahreszeit (Sommer, Übergangszeit, Winter).

Die Jahreszeit wird anhand der durchschnittlichen Tagestemperatur \( T_{\text{avg}} \) folgendermaßen bestimmt:
\[
\text{Saison} = 
\begin{cases} 
\text{Winter (W)} & \text{wenn } T_{\text{avg}} < 5^\circ\text{C} \\
\text{Übergangszeit (Ü)} & \text{wenn } 5^\circ\text{C} \leq T_{\text{avg}} \leq 15^\circ\text{C} \\
\text{Sommer (S)} & \text{wenn } T_{\text{avg}} > 15^\circ\text{C} 
\end{cases}
\]

Jeder Tag wird als Wochentag oder Wochenende/Feiertag klassifiziert, was zu einem kombinierten Profiltag führt, z.B. "WSH" (Winter, Wochentag, hoher Bewölkungsgrad).

\subsection{Berechnungsansatz nach VDI 4655}

\subsubsection{Jahresenergieverbrauch und Aufteilung auf tägliche Profile}
Der Jahresenergieverbrauch (JEV) wird für Heizung, Warmwasser und Strom separat angegeben. Dieser wird auf die Tage des Jahres verteilt, wobei tages-, saison- und klimazonenspezifische Faktoren berücksichtigt werden. 

Die Tagesbedarfe für Heizung und Warmwasser werden folgendermaßen berechnet:
\[
Q_{\text{Tag, Heizung}} = JEV_{\text{Heizung}} \cdot f_{\text{Heizung, TT}} 
\]
\[
Q_{\text{Tag, WW}} = JEV_{\text{WW}} \cdot f_{\text{WW, TT}} 
\]
wobei:
\begin{itemize}
    \item \( f_{\text{Heizung, TT}} \) und \( f_{\text{WW, TT}} \) spezifische Tagesfaktoren sind, die den Einfluss der Saison, des Tages und des Klimas berücksichtigen.
\end{itemize}

\subsubsection{Berechnung stündlicher und viertelstündlicher Lastprofile}
Nachdem der tägliche Energiebedarf ermittelt wurde, wird dieser auf stündliche und viertelstündliche Intervalle verteilt. Die Aufteilung erfolgt auf Basis der in der VDI 4655 definierten Standardlastprofile, die typische Nutzungszyklen im Tagesverlauf widerspiegeln. Dabei werden für jede Viertelstunde des Tages spezifische Lastfaktoren verwendet.

Für die viertelstündliche Aufteilung wird der Tagesbedarf \( Q_{\text{Tag}} \) auf 96 Viertelstunden des Tages verteilt:
\[
Q_{\text{15min, Heizung}} = Q_{\text{Tag, Heizung}} \cdot f_{\text{15min, Heizung}}
\]
\[
Q_{\text{15min, WW}} = Q_{\text{Tag, WW}} \cdot f_{\text{15min, WW}}
\]
\noindent
Hierbei ist \( f_{\text{15min}} \) der Lastfaktor, der für jede Viertelstunde eines Tages gilt.

\subsubsection{Korrektur der Lastprofile basierend auf tatsächlichem Verbrauch}
Der tatsächliche Energieverbrauch kann von den Standardwerten der VDI 4655 abweichen. In diesem Fall erfolgt eine Korrektur der normierten viertelstündlichen Profile. Der korrigierte viertelstündliche Bedarf wird folgendermaßen berechnet:
\[
Q_{\text{15min, korr}} = \frac{Q_{\text{15min, norm}}}{\sum Q_{\text{15min, norm}}} \cdot JEV
\]
Hierbei wird der normierte viertelstündliche Bedarf \( Q_{\text{15min, norm}} \) so skaliert, dass er den tatsächlichen Jahresenergieverbrauch \( JEV \) berücksichtigt.

\subsection{Anwendungsbereiche des VDI 4655-Profils}
Das VDI 4655-Verfahren wird vor allem zur Simulation und Modellierung von Energieverbrauchsprofilen in Wohngebäuden verwendet. Anwendungsbereiche sind:

\begin{itemize}
    \item \textbf{Simulation von Lastprofilen}: Ermöglicht eine detaillierte Simulation des stündlichen oder viertelstündlichen Energiebedarfs für Heizung, Warmwasser und Strom.
    \item \textbf{Netzplanung und Dimensionierung}: Hilft bei der Planung und Dimensionierung von Heiz- und Stromnetzen, insbesondere in Fernwärme- oder Stromversorgungsnetzen.
    \item \textbf{Optimierung der Energienutzung}: Liefert eine Basis für die Optimierung der Energienutzung und die Integration erneuerbarer Energien.
\end{itemize}

\subsection{Zusammenfassung}
Das VDI 4655-Verfahren bietet eine strukturierte Methode zur Berechnung von detaillierten Energieverbrauchsprofilen für Heizung, Warmwasser und Strom. Durch die Berücksichtigung von Wetterdaten, Gebäudetypen und typischen Tages- und Jahresprofilen ermöglicht es eine realitätsnahe Simulation des Energiebedarfs von Wohngebäuden.

Das Verfahren ermöglicht die Erstellung viertelstündlicher Profile, die zur Netzplanung und zur Optimierung der Energieversorgung verwendet werden können. Besonders hervorzuheben ist die Anpassbarkeit des Verfahrens an unterschiedliche Klimazonen und Gebäudetypen, was es für eine Vielzahl von Anwendungsszenarien geeignet macht.

\section{Berechnungsmethode: BDEW}
\label{sec:bdew}

Das BDEW-Standardlastprofilverfahren (SLP) ist eine weit verbreitete Methode zur Berechnung des stündlichen Wärmebedarfs eines Gebäudes basierend auf Jahresenergieverbrauch und Wetterbedingungen. Es wird verwendet, um typische Lastprofile für die Heizwärme (Raumwärme) und den Warmwasserbedarf eines Gebäudes zu erstellen. Dies ermöglicht eine detaillierte Simulation von Energieverbrauchsprofilen, die in der Energiewirtschaft genutzt werden können, um z.B. die Planung und Steuerung von Wärmenetzen zu optimieren.

Die Berechnung basiert auf einer Kombination von physikalischen und statistischen Modellen, die temperaturabhängige Profile und Nutzungsfaktoren über den Tages-, Wochen- und Jahresverlauf berücksichtigen. Das Ziel ist es, den gesamten jährlichen Wärmebedarf auf stündlicher Basis realitätsgetreu zu modellieren.

\subsection{Grundlegende Komponenten der Wärmebedarfsberechnung}
Der Gesamtwärmebedarf eines Gebäudes setzt sich aus zwei wesentlichen Komponenten zusammen:

\begin{itemize}
    \item \textbf{Heizwärmebedarf (HWB)}: Die Energie, die benötigt wird, um die Raumtemperatur auf einem gewünschten Niveau zu halten. Diese hängt stark von der Außentemperatur, der Gebäudedämmung und den internen Wärmelasten ab.
    \item \textbf{Warmwasserbedarf (WWB)}: Der Energiebedarf für die Erwärmung des Brauchwassers für Haushaltszwecke. Dieser Bedarf ist im Gegensatz zum Heizwärmebedarf weitgehend unabhängig von der Außentemperatur, wird aber durch das Nutzungsverhalten bestimmt.
\end{itemize}

Das BDEW-SLP-Verfahren nutzt verschiedene Koeffizienten und Faktoren, um den Einfluss dieser beiden Komponenten auf den täglichen und stündlichen Wärmebedarf zu modellieren.

\subsection{Jahreswärmebedarf und Tagesprofile}

\subsubsection{Ausgangspunkt: Der Jahreswärmebedarf (JWB)}
Der Jahreswärmebedarf eines Gebäudes wird meist in Kilowattstunden (kWh) angegeben und beschreibt den gesamten Energieverbrauch für Heizung und Warmwasser über ein Jahr. Dieser Wert wird in der Praxis z.B. durch Abrechnungsdaten oder Messungen ermittelt und stellt die Grundlage der weiteren Berechnungen dar.

\subsubsection{Aufteilung in tägliche Profile}
Der erste Schritt besteht darin, den Jahreswärmebedarf auf die einzelnen Tage des Jahres aufzuteilen. Diese Aufteilung erfolgt auf Basis der täglichen Temperaturdaten und der spezifischen Tagesprofile, die vom BDEW vorgegeben werden. Das Tagesprofil bestimmt, wie der Wärmebedarf an einem bestimmten Tag (z.B. ein Montag im Januar) aussieht.

Für jeden Tag wird der Heizwärmebedarf folgendermaßen berechnet:
\[
Q_{\text{Tag, Heizung}} = f_{\text{Tagesprofil}} \cdot F_{\text{Tagesfaktor}} \cdot m_H \cdot T_{\text{avg}} + b_H
\]
\noindent
Hierbei sind:
\begin{itemize}
    \item \( f_{\text{Tagesprofil}} \): Ein spezifischer Koeffizient, der das Heizverhalten an einem bestimmten Tag beschreibt.
    \item \( F_{\text{Tagesfaktor}} \): Ein tagesabhängiger Faktor, der den Einfluss des Wochentages oder Feiertags auf den Wärmebedarf darstellt.
    \item \( m_H \) und \( b_H \): Lineare Koeffizienten, die den Temperaturverlauf über den Tag hinweg berücksichtigen.
    \item \( T_{\text{avg}} \): Die Tagesdurchschnittstemperatur.
\end{itemize}

\subsubsection{Temperaturabhängige Berechnung des Heizwärmebedarfs}
Der Heizwärmebedarf ist eng an die Außentemperatur gekoppelt. Bei niedrigeren Außentemperaturen muss mehr Energie für das Heizen aufgewendet werden, um die Raumtemperatur konstant zu halten. Der Heizwärmebedarf wird durch eine temperaturabhängige Funktion modelliert:
\[
Q_{\text{Heizung}}(T) = \frac{A}{1 + \left( \frac{B}{T_{\text{ref}} - 40} \right)^C} + m_H \cdot T_{\text{avg}} + b_H
\]
wobei:
\begin{itemize}
    \item \( A \), \( B \), und \( C \) profiltypische Koeffizienten sind, die das spezifische Heizverhalten des Gebäudes definieren.
    \item \( T_{\text{ref}} \) ist die Referenztemperatur, die aus den stündlichen Temperaturdaten berechnet wird.
\end{itemize}

Diese Funktion stellt sicher, dass bei extrem niedrigen Außentemperaturen der Heizbedarf stark ansteigt, während er bei höheren Temperaturen entsprechend abnimmt.

\subsubsection{Berechnung des Warmwasserbedarfs}
Der Warmwasserbedarf wird durch eine ähnliche Gleichung wie der Heizwärmebedarf modelliert, jedoch ist er weniger stark von der Außentemperatur abhängig. Die Berechnung erfolgt über die Gleichung:
\[
Q_{\text{WW}}(T) = m_W \cdot T_{\text{avg}} + b_W
\]
wobei:
\begin{itemize}
    \item \( m_W \) und \( b_W \) lineare Koeffizienten für den Warmwasserbedarf sind.
    \item \( T_{\text{avg}} \) die Tagesdurchschnittstemperatur ist.
\end{itemize}
Für das Warmwasser ist die Temperaturabhängigkeit weniger relevant, da der Warmwasserbedarf eher durch das Nutzungsverhalten (z.B. morgendliches Duschen) bestimmt wird.

\subsection{Tages- und Wochenfaktoren}
Neben den Temperaturabhängigkeiten werden auch tages- und wochenabhängige Faktoren in die Berechnung einbezogen. Diese Faktoren spiegeln das typische Verbrauchsverhalten an verschiedenen Wochentagen wider. So ist der Wärmebedarf an einem Montag anders als an einem Sonntag, da der Montag typischerweise ein Arbeitstag ist und andere Heizmuster vorliegen.

Die tagesabhängigen Lastprofile werden über Faktoren \( F_{\text{Tag}} \) angepasst:
\[
F_{\text{Tag}} = F_{\text{Wochentag}} \cdot F_{\text{Wochenfaktor}} \cdot F_{\text{Feiertag}}
\]
Diese Faktoren berücksichtigen z.B. die geringere Nutzung an Wochenenden oder Feiertagen und reduzieren den berechneten Heiz- oder Warmwasserbedarf entsprechend.

\subsection{Berechnung stündlicher Lastprofile}

\subsubsection{Aufteilung des Tageswärmebedarfs auf Stunden}
Nachdem der Tageswärmebedarf für Heizung und Warmwasser berechnet wurde, wird dieser Bedarf auf die Stunden des Tages verteilt. Hierbei wird das typische Nutzungsverhalten im Tagesverlauf berücksichtigt, indem stündliche Koeffizienten \( f_{\text{Stunde}} \) verwendet werden. Diese Koeffizienten geben an, welcher Anteil des Tagesbedarfs in einer bestimmten Stunde auftritt.

Der stündliche Wärmebedarf wird mit der folgenden Interpolationsformel berechnet:
\[
Q_{\text{Heizung, Stunde}} = Q_{\text{Tag, Heizung}} \cdot \left( f_{\text{Stunde}} + \frac{T_{\text{aktuell}} - T_{\text{Grenze}}}{5} \cdot (f_{\text{Stunde, T1}} - f_{\text{Stunde, T2}}) \right)
\]
Hierbei wird zwischen zwei Temperaturgrenzwerten \( T_{\text{Grenze}} \) interpoliert, um einen fließenden Übergang zwischen den stündlichen Lasten zu gewährleisten.

\subsubsection{Stündliche Profile für Heizung und Warmwasser}
Der Wärmebedarf wird für jede Stunde des Tages sowohl für die Heizung als auch für den Warmwasserbedarf berechnet. Diese stündlichen Profile sind wichtig, um den Verbrauch über den Tag hinweg detailliert abzubilden. Insbesondere bei stark schwankenden Außentemperaturen ergeben sich deutliche Unterschiede im stündlichen Heizwärmebedarf.

\subsection{Anpassung des Warmwasseranteils am Gesamtwärmebedarf}
Falls der tatsächliche Warmwasseranteil bekannt ist, kann dieser im Modell berücksichtigt werden. Der initiale Warmwasseranteil wird als Verhältnis des berechneten Warmwasserbedarfs zum gesamten Wärmebedarf ermittelt:
\[
\text{WW-Anteil} = \frac{Q_{\text{WW}}}{Q_{\text{Heizung}} + Q_{\text{WW}}}
\]
Falls der tatsächliche Warmwasseranteil vom berechneten abweicht, kann dieser durch einen Korrekturfaktor angepasst werden. Die Berechnung erfolgt durch Skalierung des berechneten Warmwasser- und Heizwärmebedarfs mit entsprechenden Korrekturfaktoren.

\subsection{Zusammenfassung}
Das BDEW-Lastprofilverfahren bietet eine detaillierte Methode zur Berechnung des stündlichen Wärmebedarfs auf Basis von Jahresverbrauchsdaten und Wetterdaten. Es berücksichtigt sowohl die Temperaturabhängigkeit des Heizbedarfs als auch tageszeitliche und wochenabhängige Faktoren. Das Verfahren eignet sich hervorragend zur Modellierung von Wärmeverbrauchsprofilen, die für die Steuerung und Optimierung von Wärmenetzen genutzt werden können.

Durch die Berücksichtigung von tages- und stundenbasierten Faktoren sowie der Anpassung an reale Verbrauchsdaten (z.B. durch den Warmwasseranteil) liefert das Verfahren genaue und praxisnahe Ergebnisse. Es ermöglicht eine präzise Abschätzung des stündlichen Wärmebedarfs für unterschiedliche Gebäudetypen und Nutzungsverhalten.



\end{document}
