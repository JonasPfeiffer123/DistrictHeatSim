\section{Berechnungsfunktion für den Erzeugermix}
\label{sec:optimize_mix_doc}

\subsection{Einleitung}
Die Optimierungsfunktion \texttt{optimize\_mix} verwendet mathematische Optimierungstechniken, um den Mix aus Energieerzeugungstechnologien zu optimieren. Das Ziel der Optimierung ist es, die Kosten, CO2-Emissionen und den Primärenergieverbrauch zu minimieren, indem verschiedene Technologien mit unterschiedlichen Parametern berücksichtigt werden.

\subsection{Mathematisches Modell}
\subsubsection{Zielgrößen}
Die Optimierung basiert auf der Minimierung einer gewichteten Summe von drei Zielgrößen:

\[
\text{Ziel} = w_{\text{WGK}} \cdot \text{WGK\_Gesamt} + w_{\text{CO2}} \cdot \text{CO2\_Emissionen\_Gesamt} + w_{\text{Primärenergie}} \cdot \text{Primärenergie\_Faktor\_Gesamt}
\]

Hierbei sind \( w_{\text{WGK}}, w_{\text{CO2}}, w_{\text{Primärenergie}} \) die Gewichte, die den Einfluss der jeweiligen Zielgröße auf das Gesamtergebnis steuern.


\subsubsection{Optimierungsverfahren}
Die Optimierung erfolgt mittels des \texttt{SLSQP}-Algorithmus, der für nichtlineare Probleme mit Nebenbedingungen geeignet ist. Der Algorithmus sucht nach den optimalen Parametern für die Technologien (z.B. Fläche für Solarthermie, Leistung für BHKW), die die gewichtete Summe der Zielgrößen minimieren.

\subsubsection{Nebenbedingungen}
Für jede Technologie werden Schranken (\emph{bounds}) für die zu optimierenden Parameter definiert, um physikalisch sinnvolle Werte sicherzustellen. Zum Beispiel:
\begin{itemize}
    \item Für die Fläche eines Solarthermie-Systems: \( \text{min\_area} \leq \text{Fläche} \leq \text{max\_area} \)
    \item Für die Leistung eines BHKW: \( \text{min\_Leistung} \leq \text{Leistung} \leq \text{max\_Leistung} \)
\end{itemize}

\subsection{Ergebnis}
Nach erfolgreicher Optimierung gibt die Funktion die optimierten Parameter für jede Technologie zurück. Diese Parameter minimieren die gewichtete Summe der Kosten, CO2-Emissionen und des Primärenergieverbrauchs.

\subsection{Zusammenfassung}
Die Funktion \texttt{optimize\_mix} erlaubt eine simultane Optimierung mehrerer Technologien basierend auf benutzerdefinierten Zielgrößen. Durch die Verwendung von mathematischen Optimierungsverfahren wie \texttt{SLSQP} werden die besten Kombinationen von Technologien und Parametern ermittelt.
