\section{WasteHeatPump Klasse}
Die \texttt{WasteHeatPump}-Klasse modelliert ein Wärmepumpensystem zur Rückgewinnung von Abwärme und erbt von der \texttt{HeatPump}-Basisklasse. Sie enthält Methoden zur Simulation der Leistung der Wärmepumpe sowie zur Berechnung verschiedener ökonomischer und ökologischer Kennzahlen auf Basis der Abwärmenutzung.

\subsection{Attribute}
\begin{itemize}
    \item \texttt{Kühlleistung\_Abwärme (float)}: Kühlleistung der Abwärmepumpe in kW.
    \item \texttt{Temperatur\_Abwärme (float)}: Temperatur der Abwärmequelle in Grad Celsius.
    \item \texttt{spez\_Investitionskosten\_Abwärme (float)}: Spezifische Investitionskosten der Abwärmepumpe pro kW. Standardwert: 500 €/kW.
    \item \texttt{spezifische\_Investitionskosten\_WP (float)}: Spezifische Investitionskosten der Wärmepumpe pro kW. Standardwert: 1000 €/kW.
    \item \texttt{min\_Teillast (float)}: Minimale Teillast als Anteil der Nennlast. Standardwert: 0,2.
    \item \texttt{co2\_factor\_electricity (float)}: CO$_2$-Faktor für den Stromverbrauch in tCO$_2$/MWh. Standardwert: 0,4 tCO$_2$/MWh.
    \item \texttt{primärenergiefaktor (float)}: Primärenergiefaktor für den Stromverbrauch. Standardwert: 2,4.
\end{itemize}

\subsection{Methoden}
\begin{itemize}
    \item \texttt{Berechnung\_WP(VLT\_L, COP\_data)}: Berechnet die Wärmelast, den Stromverbrauch und die angepassten Vorlauftemperaturen für die Abwärmepumpe.
    \begin{itemize}
        \item \textbf{VLT\_L (array-like)}: Vorlauftemperaturen in Grad Celsius.
        \item \textbf{COP\_data (array-like)}: COP-Daten zur Interpolation.
    \end{itemize}
    Gibt die Wärmelast und den Stromverbrauch für die Abwärmepumpe zurück.

    \item \texttt{abwärme(Last\_L, VLT\_L, COP\_data, duration)}: Berechnet die Abwärme und weitere Leistungskennzahlen für die Wärmepumpe.
    \begin{itemize}
        \item \textbf{Last\_L (array-like)}: Lastanforderung in kW.
        \item \textbf{VLT\_L (array-like)}: Vorlauftemperaturen in Grad Celsius.
        \item \textbf{COP\_data (array-like)}: COP-Daten zur Leistungsberechnung.
        \item \textbf{duration (float)}: Dauer des Zeitschritts in Stunden.
    \end{itemize}
    Gibt die erzeugte Wärmemenge, den Strombedarf, die Wärmeleistung und die elektrische Leistung zurück.

    \item \texttt{calculate(VLT\_L, COP\_data, Strompreis, q, r, T, BEW, stundensatz, duration, general\_results)}: 
    Berechnet die ökonomischen und ökologischen Kennzahlen für die Abwärmepumpe.
    \begin{itemize}
        \item \textbf{VLT\_L (array-like)}: Vorlauftemperaturen in Grad Celsius.
        \item \textbf{COP\_data (array-like)}: COP-Daten zur Leistungsberechnung.
        \item \textbf{Strompreis (float)}: Strompreis in €/MWh.
        \item \textbf{q (float)}: Kapitalrückgewinnungsfaktor.
        \item \textbf{r (float)}: Preissteigerungsfaktor.
        \item \textbf{T (int)}: Betrachtungszeitraum in Jahren.
        \item \textbf{BEW (float)}: Abzinsungsfaktor für Betriebskosten.
        \item \textbf{stundensatz (float)}: Arbeitskosten pro Stunde in €/Stunde.
        \item \textbf{duration (float)}: Dauer jedes Simulationsschritts in Stunden.
        \item \textbf{general\_results (dict)}: Allgemeine Ergebnisse, inklusive Lastprofil.
    \end{itemize}
    Gibt ein Wörterbuch mit den berechneten Kennzahlen, einschließlich Wärmemenge, Strombedarf, CO$_2$-Emissionen und Primärenergieverbrauch, zurück.

    \item \texttt{to\_dict()}: Wandelt die Objektattribute in ein Wörterbuch um.
    
    \item \texttt{from\_dict(data)}: Erstellt ein Objekt aus einem Wörterbuch von Attributen.
\end{itemize}

\subsection{Ökonomische und ökologische Überlegungen}
Die \texttt{WasteHeatPump}-Klasse berechnet die gewichteten Durchschnittskosten der Wärmeerzeugung (WGK) für die Abwärmepumpe, die die Installations-, Betriebs- und Stromkosten berücksichtigen. Die Klasse berechnet auch die spezifischen CO$_2$-Emissionen basierend auf dem Stromverbrauch sowie den Primärenergieverbrauch des Systems.

\subsection{Nutzungsbeispiel}
Das folgende Beispiel zeigt, wie die \texttt{WasteHeatPump}-Klasse initialisiert und verwendet wird, um die Leistung eines Abwärmenutzungssystems zu berechnen:

\begin{verbatim}
waste_heat_pump = WasteHeatPump(
    name="Waste Heat Pump System",
    Kühlleistung_Abwärme=100,  # kW
    Temperatur_Abwärme=60  # °C
)
results = waste_heat_pump.calculate(
    VLT_L=temperature_profile, 
    COP_data=cop_profile, 
    Strompreis=150,  # €/MWh
    q=0.05, r=0.02, T=20, 
    BEW=0.85, 
    stundensatz=45, 
    duration=1, 
    general_results=load_profile
)
\end{verbatim}
In diesem Beispiel wird ein Abwärmenutzungssystem mit einer Kühlleistung von 100 kW und einer Abwärmequellentemperatur von 60°C erstellt. Die Leistungskennzahlen und wirtschaftlichen Bewertungen werden basierend auf den Eingabedaten berechnet.
