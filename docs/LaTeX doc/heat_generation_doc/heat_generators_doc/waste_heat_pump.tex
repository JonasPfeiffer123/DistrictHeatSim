\section{WasteHeatPump Class}
The \texttt{WasteHeatPump} class models a waste heat recovery heat pump system, inheriting from the \texttt{HeatPump} base class. It includes methods for simulating the performance of the heat pump and calculating various economic and environmental metrics based on waste heat recovery.

\subsection{Attributes}
\begin{itemize}
    \item \texttt{Kühlleistung\_Abwärme (float)}: Cooling capacity of the waste heat pump in kW.
    \item \texttt{Temperatur\_Abwärme (float)}: Temperature of the waste heat source in degrees Celsius.
    \item \texttt{spez\_Investitionskosten\_Abwärme (float)}: Specific investment costs for the waste heat pump per kW. Default is 500 €/kW.
    \item \texttt{spezifische\_Investitionskosten\_WP (float)}: Specific investment costs of the heat pump per kW. Default is 1000 €/kW.
    \item \texttt{min\_Teillast (float)}: Minimum partial load as a fraction of nominal load. Default is 0.2.
    \item \texttt{co2\_factor\_electricity (float)}: CO$_2$ factor for electricity consumption, in tCO$_2$/MWh. Default is 0.4 tCO$_2$/MWh.
    \item \texttt{primärenergiefaktor (float)}: Primary energy factor for electricity. Default is 2.4.
\end{itemize}

\subsection{Methods}
\begin{itemize}
    \item \texttt{Berechnung\_WP(VLT\_L, COP\_data)}: Calculates the heat load, electric power consumption, and adjusted flow temperatures for the waste heat pump.
    \begin{itemize}
        \item \textbf{VLT\_L (array-like)}: Flow temperatures in degrees Celsius.
        \item \textbf{COP\_data (array-like)}: Coefficient of performance (COP) data for interpolation.
    \end{itemize}
    Returns the heat load and electric power consumption for the waste heat pump.

    \item \texttt{abwärme(Last\_L, VLT\_L, COP\_data, duration)}: Calculates the waste heat and other performance metrics for the heat pump.
    \begin{itemize}
        \item \textbf{Last\_L (array-like)}: Load demand in kW.
        \item \textbf{VLT\_L (array-like)}: Flow temperatures in degrees Celsius.
        \item \textbf{COP\_data (array-like)}: COP data for performance calculation.
        \item \textbf{duration (float)}: Duration of each time step in hours.
    \end{itemize}
    Returns the heat energy produced, electricity demand, heat output, and electric power output.

    \item \texttt{calculate(VLT\_L, COP\_data, Strompreis, q, r, T, BEW, stundensatz, duration, general\_results)}: 
    Calculates the economic and environmental metrics for the waste heat pump system.
    \begin{itemize}
        \item \textbf{VLT\_L (array-like)}: Flow temperatures in degrees Celsius.
        \item \textbf{COP\_data (array-like)}: COP data for performance calculation.
        \item \textbf{Strompreis (float)}: Price of electricity in €/MWh.
        \item \textbf{q (float)}: Capital recovery factor.
        \item \textbf{r (float)}: Price escalation factor.
        \item \textbf{T (int)}: Consideration period in years.
        \item \textbf{BEW (float)}: Discount rate for operational costs.
        \item \textbf{stundensatz (float)}: Hourly labor rate in €/hour.
        \item \textbf{duration (float)}: Time step duration in hours.
        \item \textbf{general\_results (dict)}: General results dictionary, including the load demand profile.
    \end{itemize}
    Returns a dictionary of the calculated metrics, including heat output, electricity demand, CO$_2$ emissions, and primary energy consumption.

    \item \texttt{to\_dict()}: Converts the object attributes into a dictionary for serialization.
    
    \item \texttt{from\_dict(data)}: Creates an object from a dictionary of attributes.
\end{itemize}

\subsection{Economic and Environmental Considerations}
The \texttt{WasteHeatPump} class calculates the weighted average cost of heat generation (WGK) for the waste heat pump, which accounts for the costs of installation, operation, and electricity consumption. The class also computes specific CO$_2$ emissions based on electricity usage, as well as the primary energy consumption of the system.

\subsection{Usage Example}
The following example demonstrates how to initialize and use the \texttt{WasteHeatPump} class to calculate the performance of a waste heat recovery system:

\begin{verbatim}
waste_heat_pump = WasteHeatPump(
    name="Waste Heat Pump System",
    Kühlleistung_Abwärme=100,  # kW
    Temperatur_Abwärme=60  # °C
)
results = waste_heat_pump.calculate(
    VLT_L=temperature_profile, 
    COP_data=cop_profile, 
    Strompreis=150,  # €/MWh
    q=0.05, r=0.02, T=20, 
    BEW=0.85, 
    stundensatz=45, 
    duration=1, 
    general_results=load_profile
)
\end{verbatim}
This example creates a waste heat recovery system with a cooling capacity of 100 kW and a waste heat source temperature of 60°C. The performance metrics and economic evaluations are calculated based on the input data.