\section{HeatPump Klasse}
Die \texttt{HeatPump}-Klasse repräsentiert ein Wärmepumpensystem und bietet Methoden zur Berechnung verschiedener Leistungs- und Wirtschaftlichkeitskennzahlen. Die Klasse ist modular aufgebaut und ermöglicht die Anpassung an unterschiedliche Wärmequellen und Anwendungsfälle. Nachfolgend werden die wichtigsten Attribute und Methoden der Klasse detailliert beschrieben.

\subsection{Attribute}
\begin{itemize}
    \item \texttt{name (str)}: Der Name der Wärmepumpe.
    \item \texttt{spezifische\_Investitionskosten\_WP (float)}: Spezifische Investitionskosten der Wärmepumpe pro kW. Standardwert: 1000 €/kW.
    \item \texttt{Nutzungsdauer\_WP (int)}: Nutzungsdauer der Wärmepumpe in Jahren. Standardwert: 20 Jahre.
    \item \texttt{f\_Inst\_WP (float)}: Installationsfaktor für die Wärmepumpe. Standardwert: 1.
    \item \texttt{f\_W\_Insp\_WP (float)}: Wartungs- und Inspektionsfaktor für die Wärmepumpe. Standardwert: 1.5.
    \item \texttt{Bedienaufwand\_WP (float)}: Betriebsaufwand für die Wärmepumpe in Stunden. Standardwert: 0.
    \item \texttt{f\_Inst\_WQ (float)}: Installationsfaktor für die Wärmequelle. Standardwert: 0.5.
    \item \texttt{f\_W\_Insp\_WQ (float)}: Wartungs- und Inspektionsfaktor für die Wärmequelle. Standardwert: 0.5.
    \item \texttt{Bedienaufwand\_WQ (float)}: Betriebsaufwand für die Wärmequelle in Stunden. Standardwert: 0.
    \item \texttt{Nutzungsdauer\_WQ\_dict (dict)}: Wörterbuch, das die Nutzungsdauer verschiedener Wärmequellen (z.B. Abwärme, Flusswasser) enthält.
    \item \texttt{co2\_factor\_electricity (float)}: CO$_2$-Emissionsfaktor für Strom in tCO$_2$/MWh. Standardwert: 2.4 tCO$_2$/MWh.
\end{itemize}

\subsection{Methoden}
\begin{itemize}
    \item \texttt{COP\_WP(VLT\_L, QT, COP\_data)}: Berechnet die Leistungszahl (COP) der Wärmepumpe, indem die COP-Daten basierend auf Vorlauftemperaturen (\texttt{VLT\_L}) und Quellentemperaturen (\texttt{QT}) interpoliert werden.
    
    Diese Methode verwendet eine zweidimensionale Interpolation basierend auf vorgegebenen Vorlauf- und Quellentemperaturen. Der COP wird für jede Kombination von Vorlauftemperatur und Quellentemperatur bestimmt. Der Interpolationsalgorithmus verwendet Gitterdaten aus einer Datei oder einem Dataset, in dem die Kennlinien der Wärmepumpe enthalten sind:
    \[
    COP = f(\texttt{VLT\_L}, \texttt{QT})
    \]
    Wo \( f \) die Interpolationsfunktion ist, die die Kennlinien der Wärmepumpe verwendet, um den entsprechenden COP zu berechnen.
    
    \item \texttt{WGK(Wärmeleistung, Wärmemenge, Strombedarf, spez\_Investitionskosten\_WQ, Strompreis, q, r, T, BEW, stundensatz)}: Berechnet die gewichteten Durchschnittskosten der Wärmeerzeugung (WGK) der Wärmepumpe auf Basis der thermischen Leistung, der Investitionskosten und der Betriebskosten.
    
    \begin{itemize}
        \item \textbf{Wärmeleistung (float)}: Erzeugte Wärmeleistung in kW.
        \item \textbf{Wärmemenge (float)}: Gesamte Wärmemenge, die von der Wärmepumpe produziert wurde, in MWh.
        \item \textbf{Strombedarf (float)}: Strombedarf der Wärmepumpe in MWh.
        \item \textbf{spez\_Investitionskosten\_WQ (float)}: Spezifische Investitionskosten für die Wärmequelle.
        \item \textbf{Strompreis (float)}: Strompreis in €/MWh.
        \item \textbf{q (float)}: Kapitalrückgewinnungsfaktor.
        \item \textbf{r (float)}: Preissteigerungsfaktor.
        \item \textbf{T (int)}: Betrachtungszeitraum in Jahren.
        \item \textbf{BEW (float)}: Abzinsungsfaktor.
        \item \textbf{stundensatz (float)}: Arbeitskosten pro Stunde in €/Stunde.
    \end{itemize}
    
    Diese Methode berechnet die Gesamtkosten der Wärmeerzeugung, indem die Investitionskosten der Wärmepumpe und der Wärmequelle über die Lebensdauer des Systems mit den Betriebskosten kombiniert werden. Die jährlichen Kosten werden mit dem Annuitätenfaktor berechnet:
    \[
    E1\_WP = \frac{ \texttt{Investitionskosten\_WP} + \texttt{Betriebskosten} }{\texttt{Wärmemenge}}
    \]
    und die spezifischen Wärmeerzeugungskosten (\texttt{WGK}) werden folgendermaßen berechnet:
    \[
    \texttt{WGK\_Gesamt} = \frac{\texttt{E1\_WP} + \texttt{E1\_WQ}}{\texttt{Wärmemenge}}
    \]
\end{itemize}

\subsection{Nutzung der Methoden}
\textbf{Beispiel zur Berechnung des COP und der Wärmeerzeugungskosten (WGK)}:

\begin{verbatim}
heat_pump = HeatPump(name="Luft-Wärmepumpe", spezifische_Investitionskosten_WP=1200)

# COP-Berechnung
VLT_L = np.array([40, 50, 60])
QT = 10
COP_data = np.array([[0, 35, 45, 55, 65],
                     [5, 3.6, 3.4, 3.2, 3.0, 2.8],
                     [10, 4.0, 3.8, 3.6, 3.4, 3.2],
                     [15, 4.4, 4.2, 4.0, 3.8, 3.6]])

COP_L, adjusted_VLT_L = heat_pump.COP_WP(VLT_L, QT, COP_data)

# WGK-Berechnung
Wärmeleistung = 50  # kW
Wärmemenge = 120  # MWh
Strombedarf = 40  # MWh
Strompreis = 80  # €/MWh
q = 0.03
r = 0.02
T = 20
BEW = 0.95
stundensatz = 50
spez_Investitionskosten_WQ = 600

WGK_Gesamt_a = heat_pump.WGK(Wärmeleistung, Wärmemenge, Strombedarf, spez_Investitionskosten_WQ, Strompreis, q, r, T, BEW, stundensatz)
\end{verbatim}
In diesem Beispiel wird der COP der Wärmepumpe auf Basis der Vorlauf- und Quellentemperaturen berechnet. Anschließend werden die gewichteten Durchschnittskosten der Wärmeerzeugung (WGK) unter Berücksichtigung von Investitions- und Betriebskosten der Wärmepumpe und der Wärmequelle berechnet.
