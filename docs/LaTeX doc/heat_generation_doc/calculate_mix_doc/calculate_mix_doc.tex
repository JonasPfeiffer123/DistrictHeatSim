\section{Optimierungsfunktion für den Erzeugermix}
\label{sec:calculate_mix_doc}

\subsection{Einleitung}
Die Berechnungsfunktion \texttt{Berechnung\_Erzeugermix} ermittelt die optimale Energieerzeugung für einen vorgegebenen Mix an Technologien. Ziel ist es, die Wärmeerzeugung für ein bestimmtes Lastprofil unter Einbeziehung verschiedener Kosten-, Effizienz- und Emissionsfaktoren zu berechnen.

\subsection{Mathematisches Modell}
\subsubsection{Eingangsparameter}
Die Berechnungsfunktion nimmt eine Reihe von Eingangsparametern an, die die technologischen und ökonomischen Bedingungen beschreiben. Diese beinhalten unter anderem:

\begin{itemize}
    \item \textbf{tech\_order}: Liste der zu betrachtenden Technologien.
    \item \textbf{initial\_data}: Tuple bestehend aus Zeitpunkten, Lastprofil, Vorlauf- und Rücklauftemperaturen.
    \item \textbf{Gaspreis}, \textbf{Strompreis}, \textbf{Holzpreis}: Energiekosten in €/kWh.
    \item \textbf{BEW}: Spezifische CO2-Emissionen des Strommixes in kg CO2/kWh.
    \item \textbf{kapitalzins}, \textbf{preissteigerungsrate}, \textbf{betrachtungszeitraum}: Finanzielle Parameter für die Kostenberechnung.
\end{itemize}

\subsubsection{Berechnungslogik}
Die Funktion berechnet zunächst die Jahreswärmebedarfe basierend auf dem Lastprofil \( L \) und der zeitlichen Auflösung:
\[
\text{Jahreswärmebedarf} = \frac{\sum L}{1000} \cdot \text{duration}
\]
Die Wärmebedarfsfunktion läuft über eine Schleife für jede Technologie in der \texttt{tech\_order}. Je nach Art der Technologie (Solarthermie, Abwärme, Geothermie usw.) wird ein spezifisches Berechnungsmodell angewandt.

\subsubsection{Technologiespezifische Berechnung}
Jede Technologie verwendet unterschiedliche Berechnungsmodelle:

\begin{itemize}
    \item \textbf{Solarthermie}: Berechnet den Ertrag basierend auf der Vorlauftemperatur und der solaren Einstrahlung aus dem Testreferenzjahr (TRY).
    \item \textbf{Wärmepumpen} und \textbf{Abwärme}: Verwenden den COP-Wert (\emph{Coefficient of Performance}) und Strompreis zur Ermittlung der Betriebsaufwendungen.
    \item \textbf{Blockheizkraftwerke (BHKW)}: Berücksichtigen sowohl thermische als auch elektrische Leistungen, sowie den Brennstoffverbrauch.
\end{itemize}

\subsection{Kapital- und Emissionskosten}
Neben den Betriebskosten werden auch kapitalgebundene und emissionsbasierte Kosten berechnet. Der kapitalgebundene Kostenanteil ergibt sich aus:
\[
A_{N,K} = A_0 \cdot \frac{(q - 1)}{1 - q^{-T}}
\]
wobei \( q = 1 + \text{Zinsrate} \).

Die spezifischen CO2-Emissionen werden pro erzeugte Wärmemenge berechnet:
\[
\text{CO2\_Emissionen} = \frac{\sum \text{Wärmemenge}_i \cdot \text{spec\_co2}_i}{\text{Jahreswärmebedarf}}
\]

\subsection{Zusammenfassung}
Die Funktion \texttt{Berechnung\_Erzeugermix} führt eine detaillierte Berechnung der Energieerzeugung durch, indem sie mehrere Technologien gleichzeitig berücksichtigt. Die Berechnung erfolgt basierend auf stündlichen Daten für Lastprofile, Temperaturen und Emissionen.