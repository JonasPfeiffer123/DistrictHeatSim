\section{Einleitung}
Dieses Dokument beschreibt die Funktionen in den Skripten \texttt{MST\_processing.py}, \texttt{simple\_MST.py}, und \texttt{import\_and\_create\_layers.py}. Die Skripte sind darauf ausgelegt, OSM-Daten zu verarbeiten, räumliche Netzwerke basierend auf Minimal Spanning Tree (MST)-Algorithmen zu erstellen und die Ergebnisse als GeoJSON-Dateien zu exportieren.

\section{Skript: \texttt{MST\_processing.py}}

\subsection{Übersicht}
Das Skript \texttt{MST\_processing.py} enthält Funktionen zur Nachbearbeitung von MST-Ergebnissen. Es erlaubt die Anpassung der MST-Segmente an Straßenverläufe und die Einführung von Zwischenpunkten zwischen gegebenen Punkten und den nächstgelegenen Straßen.

\subsection{Funktion: \texttt{add\_intermediate\_points(points\_gdf, street\_layer, max\_distance=200, point\_interval=10)}}
\textbf{Beschreibung:}\\
Diese Funktion fügt Zwischenpunkte zwischen gegebenen Punkten und den nächstgelegenen Straßen hinzu. Dies ist nützlich, um Netzwerke zu verfeinern und genauere Straßenanbindungen zu gewährleisten. Es werden Punkte in regelmäßigen Abständen (basierend auf \texttt{point\_interval}) entlang der Verbindung zwischen den gegebenen Punkten und den Straßen hinzugefügt.

\subsection{Funktion: \texttt{adjust\_segments\_to\_roads(mst\_gdf, street\_layer, all\_end\_points\_gdf, threshold=5)}}
\textbf{Beschreibung:}\\
Passt die MST-Segmente so an, dass sie den Straßenlinien genauer folgen. Die Funktion iteriert durch die MST-Segmente und passt die Linien an die Straßen an, wenn sie eine bestimmte Distanz (\texttt{threshold}) überschreiten.

\subsection{Funktion: \texttt{generate\_mst(points)}}
\textbf{Beschreibung:}\\
Erstellt einen Minimal Spanning Tree (MST) basierend auf einem Satz von Punkten. Die Kanten des MST werden als \texttt{LineString}-Objekte in einem \texttt{GeoDataFrame} gespeichert.

\section{Skript: \texttt{simple\_MST.py}}

\subsection{Übersicht}
Das Skript \texttt{simple\_MST.py} enthält die wesentlichen Funktionen zur Erstellung eines MST-Netzwerks und zur Verarbeitung von räumlichen Daten. Es verwendet verschiedene Algorithmen, um Netzwerke zu generieren und den Verlauf von Straßen in das Netzwerk zu integrieren.

\subsection{Funktion: \texttt{create\_offset\_points(point, distance, angle\_degrees)}}
\textbf{Beschreibung:}\\
Erzeugt einen Punkt, der um einen bestimmten Abstand und Winkel von einem gegebenen Punkt versetzt ist. Diese Funktion wird verwendet, um neue Punkte in einem festgelegten Abstand von bestehenden Punkten zu generieren.

\subsection{Funktion: \texttt{generate\_network\_fl(layer\_points\_fl, layer\_wea, street\_layer, algorithm="MST")}}
\textbf{Beschreibung:}\\
Erzeugt ein Netzwerk aus den Fließpunkten (\texttt{layer\_points\_fl}) und den Wärmeaustauscherpunkten (\texttt{layer\_wea}) unter Berücksichtigung der Straßenverläufe (\texttt{street\_layer}). Die Funktion unterstützt verschiedene Algorithmen zur Netzwerkgenerierung, einschließlich MST und A*-Algorithmus.

\subsection{Funktion: \texttt{generate\_return\_lines(layer\_points\_rl, layer\_wea, fixed\_distance\_rl, fixed\_angle\_rl, street\_layer)}}
\textbf{Beschreibung:}\\
Erstellt Rücklaufleitungen basierend auf den Rücklaufpunkten (\texttt{layer\_points\_rl}) und Wärmeaustauscherpunkten (\texttt{layer\_wea}). Die Leitungen werden versetzt und entlang der nächstgelegenen Straßen angeordnet.

\section{Skript: \texttt{import\_and\_create\_layers.py}}

\subsection{Übersicht}
Das Skript \texttt{import\_and\_create\_layers.py} dient der Verarbeitung von OSM-Daten und der Erstellung von Netzwerkschichten. Es kann Geodaten importieren, Schichten basierend auf den Daten erstellen und die Ergebnisse als GeoJSON-Dateien exportieren.

\subsection{Funktion: \texttt{import\_osm\_street\_layer(osm\_street\_layer\_geojson\_file)}}
\textbf{Beschreibung:}\\
Importiert die OSM-Straßenschicht aus einer GeoJSON-Datei und gibt sie als \texttt{GeoDataFrame} zurück. Diese Funktion ermöglicht die Verwendung von OSM-Daten für die Netzwerkgenerierung.

\subsection{Funktion: \texttt{generate\_lines(layer, distance, angle\_degrees, df=None)}}
\textbf{Beschreibung:}\\
Generiert Linien, die von den Punkten in der \texttt{GeoDataFrame} \texttt{layer} um eine bestimmte Distanz und einen bestimmten Winkel versetzt sind. Die Funktion kann zusätzlich Attribute aus einer übergebenen Datenstruktur (\texttt{df}) übernehmen.

\subsection{Funktion: \texttt{generate\_and\_export\_layers(osm\_street\_layer\_geojson\_file\_name, data\_csv\_file\_name, coordinates, base\_path, fixed\_angle=0, fixed\_distance=1, algorithm="MST")}}
\textbf{Beschreibung:}\\
Erzeugt die Netzwerkschichten basierend auf den gegebenen Daten und exportiert diese als GeoJSON-Dateien. Es verwendet verschiedene Algorithmen zur Netzwerkgenerierung, einschließlich des MST-Algorithmus.

\section{Zusammenfassung}
Die bereitgestellten Skripte bieten eine umfassende Lösung zur Erstellung von Netzwerken basierend auf räumlichen Daten und der Straßeninfrastruktur. Die Funktionen ermöglichen die flexible Anpassung der Netzwerke an Straßenverläufe, die Generierung von MST-Netzwerken sowie die einfache Verarbeitung und den Export der Ergebnisse in GeoJSON-Format.
