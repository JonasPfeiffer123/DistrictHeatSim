\section{OSM}

\subsection{Skript: \texttt{import\_osm\_data\_geojson.py}}

Dieses Skript ist darauf ausgelegt, OpenStreetMap (OSM)-Daten herunterzuladen und in das GeoJSON-Format umzuwandeln. Der Fokus liegt dabei auf bestimmten Elementen (z. B. Straßen oder Gebäuden) innerhalb eines angegebenen Stadtgebiets. Das Skript enthält Funktionen zum Erstellen einer Overpass-Abfrage, zum Herunterladen von Daten, zur Verarbeitung der Daten in GeoJSON-Features und zum Speichern der resultierenden GeoJSON-Datei.

\subsubsection{Funktion: \texttt{build\_query(city\_name, tags, element\_type="way")}}

\textbf{Beschreibung:}\\
Generiert eine Overpass-Abfrage, um OSM-Daten basierend auf festgelegten Kriterien wie Stadtname, Tags (Schlüssel-Wert-Paare) und Elementtyp abzurufen. Der Standardwert für \texttt{element\_type} ist \texttt{"way"} für Straßen.

\textbf{Rückgabewert:}\\
Gibt die erstellte Abfrage als Zeichenfolge zurück.

\subsubsection{Funktion: \texttt{download\_data(query, element\_type)}}

\textbf{Beschreibung:}\\
Führt die Overpass-Abfrage aus, um OSM-Daten abzurufen und in GeoJSON-Features zu konvertieren. Die Funktion unterstützt zwei Elementtypen: \texttt{"way"} für Straßen und \texttt{"building"} für Gebäude.

\textbf{Rückgabewert:}\\
Gibt eine \texttt{GeoJSON-FeatureCollection} zurück, die die heruntergeladenen Daten enthält.

\subsubsection{Funktion: \texttt{json\_serial(obj)}}

\textbf{Beschreibung:}\\
Ein JSON-Serialisierer, der zum Umgang mit Objekten dient, die nicht nativ in JSON serialisierbar sind. Diese Funktion konvertiert \texttt{Decimal}-Objekte in Gleitkommazahlen.

\textbf{Anwendungsfall:}\\
Wird in der Funktion \texttt{save\_to\_file} zur Serialisierung von \texttt{Decimal}-Objekten verwendet.

\subsubsection{Funktion: \texttt{save\_to\_file(geojson\_data, filename)}}

\textbf{Beschreibung:}\\
Speichert die GeoJSON-Daten in einer angegebenen Datei mit richtiger Formatierung. Verwendet die Funktion \texttt{json\_serial}, um \texttt{Decimal}-Objekte zu serialisieren.

\subsubsection{Funktion: \texttt{run\_here()}}

\textbf{Beschreibung:}\\
Definiert Standardparameter für Stadtname und Tags (z. B. Straßentyp), erstellt eine Overpass-Abfrage, lädt OSM-Daten herunter und speichert diese als GeoJSON-Datei. Diese Funktion kann entkommentiert und ausgeführt werden, um das Skript mit vordefinierten Einstellungen auszuführen.

\subsection{Hinweis}

\begin{itemize}
    \item Das Skript ermöglicht es, die Datenextraktion anzupassen, indem Stadt, OSM-Tags und der Elementtyp (\texttt{way} oder \texttt{building}) festgelegt werden.
    \item Die \texttt{run\_here()}-Funktion kann entkommentiert und ausgeführt werden, um das Skript mit vordefinierten Parametern zu starten.
\end{itemize}

\textbf{Zusammenfassung:}\\
Dieses Skript ist ein nützliches Werkzeug, um OSM-Daten abzurufen und in das GeoJSON-Format umzuwandeln. Dadurch werden die Daten für verschiedene geospatiale Anwendungen und Analysen nutzbar.
