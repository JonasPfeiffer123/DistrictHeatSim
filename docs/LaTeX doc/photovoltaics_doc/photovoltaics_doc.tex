\documentclass{article}
\usepackage{amsmath}
\usepackage{amsfonts}
\usepackage{amssymb}
\usepackage{graphicx}
\usepackage{hyperref}

\title{Berechnung der Solarstrahlung und des Photovoltaik-Ertrags}
\author{Dipl.-Ing. (FH) Jonas Pfeiffer}
\date{2024-07-31}

\begin{document}

\maketitle

\section*{1. Einleitung}
Dieses Dokument beschreibt die Berechnung der Solarstrahlung und des photovoltaischen Ertrags anhand der Funktionen aus dem Skript \texttt{photovoltaics.py}. Die Berechnung basiert auf der Analyse von Wetterdaten (Testreferenzjahr, TRY), welche Temperatur, Windgeschwindigkeit und Strahlungsintensität (direkte und diffuse Strahlung) umfassen. Die erzeugte Strahlung wird zur Berechnung der Einstrahlung auf geneigten Flächen sowie zur Bestimmung des PV-Ertrags verwendet.

\section*{2. Grundlagen der Solarstrahlungsberechnung}
Die Berechnung der Solarstrahlung auf einer geneigten Fläche basiert auf mehreren Eingangsparametern, darunter horizontale Strahlungsdaten, der Neigungswinkel des Kollektors und die geographische Lage des Standorts.

\subsection*{2.1 Tageswinkel und Zeitgleichung}
Der Tageswinkel \( B \) wird berechnet, um die Position der Erde relativ zur Sonne im Jahreszyklus zu bestimmen:
\[
B = \frac{360}{365} \times (\text{Tag\_des\_Jahres} - 1)
\]
Die Zeitgleichung \( E \) berücksichtigt die Differenz zwischen der Sonnenzeit und der Standardzeit. Diese wird mit folgender Formel berechnet:
\[
E = 229.2 \times \left( 0.000075 + 0.001868 \cos(B) - 0.032077 \sin(B) - 0.014615 \cos(2B) - 0.04089 \sin(2B) \right)
\]

\subsection*{2.2 Solarzeit und Stundenwinkel}
Die Solarzeit \( t_{\text{sol}} \) wird verwendet, um die tatsächliche Position der Sonne am Himmel zu berechnen:
\[
t_{\text{sol}} = \frac{(\text{Stunden\_des\_Tages} - 0.5) \times 3600 + E \times 60 + 4 \times (\text{STD\_Längengrad} - \text{Längengrad}) \times 60}{3600}
\]
Der Stundenwinkel \( \omega \) der Sonne wird dann folgendermaßen berechnet:
\[
\omega = -180 + t_{\text{sol}} \times \frac{180}{12}
\]

\subsection*{2.3 Solarer Zenit- und Azimutwinkel}
Der Solare Zenitwinkel \( \theta_Z \) beschreibt den Winkel zwischen der Sonne und dem Zenit:
\[
\theta_Z = \arccos\left( \cos(\phi) \cos(\delta) \cos(\omega) + \sin(\phi) \sin(\delta) \right)
\]
Der Solare Azimutwinkel \( \gamma_S \) wird wie folgt berechnet:
\[
\gamma_S = \text{sgn}(\omega) \times \arccos\left( \frac{\cos(\theta_Z) \sin(\phi) - \sin(\delta)}{\sin(\theta_Z) \cos(\phi)} \right)
\]
wobei \( \phi \) die geographische Breite und \( \delta \) die Solardeklinationswinkel ist:
\[
\delta = 23.45 \times \sin\left( \frac{360 \times (284 + \text{Tag\_des\_Jahres})}{365} \right)
\]

\subsection*{2.4 Einfallswinkel der Strahlung auf den Kollektor}
Der Einfallswinkel der Strahlung auf den Kollektor \( \theta_{\text{Einfall}} \) wird durch die Kombination von Azimut- und Zenitwinkel berechnet:
\[
\theta_{\text{Einfall}} = \arccos\left( \cos(\theta_Z) \cos(\beta) + \sin(\theta_Z) \sin(\beta) \cos(\gamma_S - \gamma_C) \right)
\]
wobei \( \beta \) der Neigungswinkel des Kollektors und \( \gamma_C \) der Azimutwinkel des Kollektors ist.

\section*{3. Berechnung der Solarstrahlung auf geneigten Flächen}
Um die Strahlungsintensität auf einer geneigten Oberfläche zu bestimmen, wird die Direktstrahlung auf eine horizontale Oberfläche (\( G_b \)) und die Diffusstrahlung (\( G_d \)) verwendet.

\subsection*{3.1 Strahlungsintensität auf der geneigten Fläche}
Die Direktstrahlung auf eine geneigte Fläche wird durch den Strahlungsfaktor \( R_b \) skaliert, der das Verhältnis der Strahlung auf die geneigte Fläche zur horizontalen Fläche darstellt:
\[
R_b = \frac{\cos(\theta_{\text{Einfall}})}{\cos(\theta_Z)}
\]
Die Gesamtstrahlung \( G_T \) auf die geneigte Fläche ergibt sich als Kombination von direkter und diffuser Strahlung sowie einer Bodenreflexion (Albedo):
\[
G_T = G_b \times R_b + G_d \times \left( \frac{1 + \cos(\beta)}{2} \right) + G_h \times \text{Albedo} \times \left( \frac{1 - \cos(\beta)}{2} \right)
\]

\section*{4. Photovoltaik-Leistungsberechnung}
Die Photovoltaik-Leistung wird auf Basis der berechneten Einstrahlung, der Fläche des PV-Systems und den spezifischen Systemparametern (z.B. Effizienz und Verluste) berechnet.

\subsection*{4.1 PV-Leistung}
Die PV-Leistung wird wie folgt berechnet:
\[
P_{\text{PV}} = G_T \times \text{Fläche} \times \eta_{\text{nom}} \times (1 - \text{Systemverluste})
\]
wobei \( G_T \) die berechnete Strahlungsintensität, \( \eta_{\text{nom}} \) der nominale Wirkungsgrad und die Systemverluste (typisch 14\%) berücksichtigt werden.

\subsection*{4.2 Einfluss der Temperatur}
Die Modultemperatur \( T_m \) wird durch die Umgebungstemperatur \( T_a \), die Strahlungsintensität und die Windgeschwindigkeit beeinflusst:
\[
T_m = T_a + \frac{G_T}{U_0 + U_1 \times \text{Windgeschwindigkeit}}
\]
Die relative Effizienz \( \eta_{\text{rel}} \) wird anhand der Irradiation \( G_1 \) und der Modultemperatur \( T_m \) mit folgender Gleichung berechnet:
\[
\eta_{\text{rel}} = 1 + k_1 \ln(G_1) + k_2 (\ln(G_1))^2 + k_3 T_1m + k_4 T_1m \ln(G_1) + k_5 T_m (\ln(G_1))^2 + k_6 T_m^2
\]

\section*{5. Zusammenfassung}
Die im Skript implementierte Methode berechnet die Solarstrahlung auf geneigte Flächen sowie den photovoltaischen Ertrag basierend auf den Parametern des Systems und den TRY-Daten. Durch die Berücksichtigung der Strahlungsintensität, der Modultemperatur und der Systemverluste wird eine realitätsnahe Schätzung des jährlichen PV-Ertrags ermöglicht. Das Verfahren eignet sich zur Modellierung von PV-Anlagen in unterschiedlichen geographischen Lagen und für eine Vielzahl von Kollektorneigungen und -ausrichtungen.

\end{document}
