\documentclass{article}
\usepackage{amsmath}
\usepackage{amsfonts}
\usepackage{amssymb}
\usepackage{graphicx}
\usepackage{hyperref}

\title{Fachliche Beschreibung des Berechnungsalgorithmus zur Berechnung der Solarstrahlung}
\author{Dipl.-Ing. (FH) Jonas Pfeiffer}
\date{2024-09-04}

\begin{document}

\maketitle

\section*{1. Einleitung}

Dieses Dokument beschreibt im Detail den Berechnungsalgorithmus zur Bestimmung der Solarstrahlung, die auf eine geneigte Oberfläche trifft. Der Algorithmus basiert auf Wetterdaten eines Testreferenzjahres (TRY) und berücksichtigt Faktoren wie den Einfallswinkel der Sonnenstrahlen, die Neigung der Oberfläche und den Albedo-Effekt. Diese Berechnungsmethode wird in der Simulation von Solaranlagen, insbesondere für solarthermische Anlagen, eingesetzt.

Die Berechnung der Solarstrahlung erfolgt mit physikalischen Modellen, die den Einfallswinkel der Strahlung auf geneigte Kollektoren berücksichtigen und die direkte, diffuse und reflektierte Strahlung berechnen.

\section*{2. Konstante und Hilfsfunktionen}

\subsection*{2.1 \texttt{DEG\_TO\_RAD}}

Die Konstante \texttt{DEG\_TO\_RAD} wird verwendet, um Winkel in Grad in Bogenmaß (Radiant) umzurechnen. Diese Konstante basiert auf der bekannten Umrechnungsformel:
\[
1^\circ = \frac{\pi}{180} \, \text{Rad}.
\]
Sie wird bei der Berechnung von trigonometrischen Funktionen verwendet, da diese Funktionen in der Programmiersprache Python Bogenmaß erwarten.

\subsection*{2.2 \texttt{deg\_to\_rad(deg)}}

Die Funktion \texttt{deg\_to\_rad(deg)} konvertiert einen Winkel in Grad in Bogenmaß. Der Algorithmus benötigt dies, da Winkel in Grad angegeben werden, aber für die Berechnungen der trigonometrischen Funktionen das Bogenmaß erforderlich ist.

\begin{verbatim}
def deg_to_rad(deg):
    return deg * DEG_TO_RAD
\end{verbatim}

\section*{3. Berechnung der Solarstrahlung}

\subsection*{3.1 \texttt{Berechnung\_Solarstrahlung}}

Diese Funktion ist der Kern des Algorithmus. Sie berechnet die auf eine geneigte Oberfläche einfallende Solarstrahlung unter Berücksichtigung direkter und diffuser Strahlung sowie reflektierter Strahlung durch Albedo.

\subsubsection*{Eingabeparameter}

Die Funktion nimmt verschiedene Eingabedaten entgegen:

\begin{itemize}
    \item \textbf{Globalstrahlung\_L}: Array der globalen Strahlungsdaten.
    \item \textbf{D\_L}: Direkte Strahlungskomponente.
    \item \textbf{Tag\_des\_Jahres\_L}: Tag des Jahres.
    \item \textbf{time\_steps}: Zeitarray.
    \item \textbf{Longitude, STD\_Longitude, Latitude}: Geografische Koordinaten der Anlage.
    \item \textbf{Albedo}: Reflektionsgrad der Umgebung.
    \item \textbf{IAM\_W, IAM\_N}: Modifikationsfaktoren für den Einstrahlungswinkel in Ost-West- bzw. Nord-Süd-Richtung.
    \item \textbf{EWCaa, CTA}: Azimutwinkel und Neigungswinkel des Solarkollektors.
\end{itemize}

\subsubsection*{Berechnungsschritte}

\textbf{1. Berechnung der Sonnenzeit und des Tagwinkels}

Der Tag des Jahres wird in einen Winkel umgerechnet, der für die Berechnung der Sonnenposition erforderlich ist. Der Tagwinkel \( B \) wird wie folgt berechnet:
\[
B = \frac{360 \cdot (\text{Tag\_des\_Jahres\_L} - 1)}{365}
\]
Zusätzlich wird eine Zeitkorrektur \( E \) berechnet, die die Abweichungen der Sonnenzeit durch die elliptische Bahn der Erde um die Sonne berücksichtigt:
\[
E = 229.2 \cdot \left( 0.000075 + 0.001868 \cos(B) - 0.032077 \sin(B) - 0.014615 \cos(2B) - 0.04089 \sin(2B) \right)
\]
Die Sonnenzeit wird unter Berücksichtigung der geographischen Länge berechnet.

\textbf{2. Berechnung des Sonnenzenitwinkels (SZA)}

Der Sonnenzenitwinkel beschreibt den Winkel zwischen dem Lot auf die Erdoberfläche und den Sonnenstrahlen. Er wird berechnet, indem die geografische Breite \( \phi \), der Stundenwinkel \( h \) und die Deklination \( \delta \) der Sonne wie folgt genutzt werden:
\[
SZA = \arccos\left( \cos(\phi) \cos(h) \cos(\delta) + \sin(\phi) \sin(\delta) \right)
\]
Der Stundenwinkel \( h \) beschreibt den Winkel der Sonne relativ zu ihrer höchsten Position am Himmel.

\textbf{3. Berechnung des Einfallswinkels auf den Kollektor}

Der Einfallswinkel \( I_aC \) der Solarstrahlung auf den geneigten Kollektor wird mit der folgenden Formel berechnet:
\[
I_aC = \arccos\left( \cos(SZA) \cos(CTA) + \sin(SZA) \sin(CTA) \cos(EWs\_az\_angle - EWCaa) \right)
\]
Hierbei ist \( CTA \) der Neigungswinkel des Kollektors und \( EWCaa \) der Azimutwinkel.

\textbf{4. Berechnung des Einstrahlungswinkels}

Für die Modifikationsfaktoren \( IAM\_W \) und \( IAM\_N \) wird der Einfallswinkel berechnet, um den Effekt der Strahlungsintensität in Ost-West- und Nord-Süd-Richtung zu berücksichtigen. Dies wird durch eine Lookup-Funktion realisiert, die den Einstrahlungswinkel in 10°-Schritten interpoliert.

\textbf{5. Berechnung der direkten und diffusen Strahlung}

Die direkte Strahlung \( Gbhoris \) auf der horizontalen Fläche wird mit dem Kosinus des Sonnenzenitwinkels multipliziert:
\[
Gbhoris = D\_L \cdot \cos(SZA)
\]
Die diffuse Strahlung ergibt sich aus der Differenz der globalen Strahlung und der direkten Strahlung:
\[
Gdhoris = Globalstrahlung\_L - Gbhoris
\]

\textbf{6. Albedo-Effekt und Gesamtstrahlung}

Der Albedo-Effekt berücksichtigt die Reflexion von Strahlung durch die Umgebung. Diese reflektierte Strahlung wird zur Berechnung der Gesamtstrahlung auf der geneigten Oberfläche hinzugefügt:
\[
GT\_H\_Gk = Gbhoris \cdot Rb + Gdhoris \cdot Ai \cdot Rb + Gdhoris \cdot (1 - Ai) \cdot 0.5 \cdot (1 + \cos(CTA)) + Globalstrahlung\_L \cdot Albedo \cdot 0.5 \cdot (1 - \cos(CTA))
\]
Hierbei beschreibt \( Rb \) das Verhältnis der Strahlungsintensität auf der geneigten Fläche zur horizontalen Fläche und \( Ai \) den atmosphärischen Diffusanteil.

\subsubsection*{Rückgabewerte}

Die Funktion liefert folgende Werte zurück:
\begin{itemize}
    \item \textbf{GT\_H\_Gk}: Die Gesamtstrahlung auf der geneigten Oberfläche.
    \item \textbf{K\_beam}: Modifizierte Strahlungsintensität.
    \item \textbf{GbT}: Direkte Strahlung auf der geneigten Fläche.
    \item \textbf{GdT\_H\_Dk}: Diffuse Strahlung auf der geneigten Fläche.
\end{itemize}

\section*{4. Zusammenfassung}

Der Algorithmus zur Berechnung der Solarstrahlung berücksichtigt die geometrische Lage des Kollektors, die Position der Sonne, atmosphärische Effekte und den Albedo-Effekt. Dies ermöglicht eine präzise Berechnung der Strahlungsenergie, die auf eine geneigte Fläche (wie einen Solarkollektor) fällt. Diese Berechnungen sind essentiell für die Simulation und Optimierung solarthermischer Anlagen in Wärmenetzen.

\end{document}
