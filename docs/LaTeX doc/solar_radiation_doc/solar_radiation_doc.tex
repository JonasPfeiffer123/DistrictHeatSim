\documentclass{article}
\usepackage{amsmath}
\usepackage{amsfonts}
\usepackage{amssymb}
\usepackage{graphicx}
\usepackage{hyperref}

\title{Fachliche Beschreibung des Berechnungsalgorithmus zur Solarstrahlungsberechnung}
\author{Dipl.-Ing. (FH) Jonas Pfeiffer}
\date{2024-09-04}

\begin{document}

\maketitle

\section*{1. Einleitung}

Dieses Dokument beschreibt detailliert den Algorithmus zur Berechnung der Solarstrahlung, die auf eine geneigte Oberfläche trifft. Der Algorithmus berücksichtigt Wetterdaten eines Testreferenzjahres (TRY), die geografische Lage, den Einfallswinkel der Sonnenstrahlen, die Neigung der Kollektorfläche und den Albedo-Effekt. Diese Methode wird zur Simulation von Solaranlagen verwendet, insbesondere für solarthermische Anwendungen in Wärmenetzen.

Yield calculation program for solar thermal energy in heating networks (calculation basis: ScenoCalc District Heating 2.0, \url{https://www.scfw.de/})

\section*{2. Grundlagen}

\subsection*{2.1 Grad-Radian-Konversion}

Die Umrechnung von Grad in Bogenmaß erfolgt durch die Konstante:
\[
\text{DEG\_TO\_RAD} = \frac{\pi}{180}
\]
Diese Konstante wird verwendet, da trigonometrische Funktionen in Python Eingaben im Bogenmaß erwarten.

Die Funktion \texttt{deg\_to\_rad(deg)} wandelt einen Winkel \( \theta \) in Grad in Bogenmaß um:
\[
\theta_{\text{rad}} = \theta_{\text{deg}} \times \frac{\pi}{180}
\]

\section*{3. Berechnung der Solarstrahlung}

Die Hauptfunktion des Algorithmus ist \texttt{Berechnung\_Solarstrahlung}, die die Strahlungsintensität auf einer geneigten Kollektorfläche berechnet. Die Berechnung erfolgt in mehreren Schritten, die im Folgenden beschrieben werden.

\subsection*{3.1 Berechnung des Tagwinkels und der Zeitgleichung}

Der Tag des Jahres \( N \) wird in einen Winkel \( B \) umgerechnet:
\[
B = \frac{360 \times (N - 1)}{365}
\]
Dieser Winkel ist notwendig, um die Sonnenposition im Jahreszyklus zu berechnen.

Die Zeitgleichung \( E \), die die Abweichungen zwischen Sonnenzeit und Standardzeit berücksichtigt, wird mit folgender Formel berechnet:
\[
E = 229.2 \cdot \left( 0.000075 + 0.001868 \cos(B) - 0.032077 \sin(B) - 0.014615 \cos(2B) - 0.04089 \sin(2B) \right)
\]

\subsection*{3.2 Berechnung der Sonnenzeit}

Die Sonnenzeit wird unter Berücksichtigung der geografischen Länge \( L \), der Standardlänge des Zeitzonenmeridians \( L_{std} \) und der Zeitgleichung \( E \) berechnet:
\[
t_{\text{solar}} = \frac{(t_{\text{Uhrzeit}} - 0.5) \cdot 3600 + E \cdot 60 + 4 \cdot (L_{std} - L) \cdot 60}{3600}
\]

\subsection*{3.3 Sonnenzenitwinkel und Deklination der Sonne}

Die Deklination der Sonne \( \delta \) wird als Funktion des Tages des Jahres berechnet:
\[
\delta = 23.45 \cdot \sin\left( \frac{360 \cdot (284 + N)}{365} \right)
\]

Der Sonnenzenitwinkel \( SZA \) beschreibt den Winkel zwischen dem Lot auf die Erdoberfläche und den Sonnenstrahlen:
\[
SZA = \arccos\left( \cos(\phi) \cos(h) \cos(\delta) + \sin(\phi) \sin(\delta) \right)
\]
wobei \( \phi \) die geografische Breite und \( h \) der Stundenwinkel der Sonne ist:
\[
h = -180 + t_{\text{solar}} \times \frac{180}{12}
\]

\subsection*{3.4 Einfallswinkel auf die geneigte Fläche}

Der Einfallswinkel \( I_aC \) der Sonnenstrahlung auf die geneigte Fläche wird berechnet, um die Intensität der direkten Strahlung auf den Kollektor zu bestimmen:
\[
I_aC = \arccos\left( \cos(SZA) \cos(CTA) + \sin(SZA) \sin(CTA) \cos(\gamma_S - \gamma_C) \right)
\]
wobei \( CTA \) der Neigungswinkel des Kollektors, \( \gamma_S \) der Sonnenazimutwinkel und \( \gamma_C \) der Azimutwinkel des Kollektors ist.

\subsection*{3.5 Berechnung der direkten und diffusen Strahlung}

Die direkte Strahlung auf die horizontale Fläche \( Gbhoris \) wird berechnet als:
\[
Gbhoris = D_L \cdot \cos(SZA)
\]

Die diffuse Strahlung \( Gdhoris \) ergibt sich als Differenz zwischen der globalen Strahlung \( G \) und der direkten Strahlung:
\[
Gdhoris = G - Gbhoris
\]

\subsection*{3.6 Atmosphärischer Diffusanteil und Gesamtstrahlung}

Der atmosphärische Diffusanteil \( Ai \) wird basierend auf der horizontalen Direktstrahlung \( Gbhoris \) und der Solarkonstanten (1367 W/m²) wie folgt berechnet:
\[
Ai = \frac{Gbhoris}{1367 \cdot (1 + 0.033 \cdot \cos(360 \cdot N / 365)) \cdot \cos(SZA)}
\]

Die Gesamtstrahlung \( GT_HGk \) auf der geneigten Oberfläche wird berechnet durch:
\[
GT_HGk = Gbhoris \cdot R_b + Gdhoris \cdot Ai \cdot R_b + Gdhoris \cdot (1 - Ai) \cdot 0.5 \cdot (1 + \cos(CTA)) + G \cdot \text{Albedo} \cdot 0.5 \cdot (1 - \cos(CTA))
\]
Hierbei beschreibt \( R_b \) das Verhältnis der Strahlungsintensität auf der geneigten Fläche zur horizontalen Fläche.

\section*{4. Modifikation der Strahlung basierend auf dem Einfallswinkel (IAM)}

Für die Anpassung der Strahlung aufgrund des Einfallswinkels wird der sogenannte Incidence Angle Modifier (IAM) verwendet. Dieser wird sowohl für die Ost-West- als auch für die Nord-Süd-Richtung berechnet. Der IAM wird durch eine Lookup-Tabelle für die Einfallswinkel interpoliert.

\section*{5. Zusammenfassung}

Dieser Algorithmus berechnet die Solarstrahlung auf geneigten Flächen, basierend auf physikalischen Modellen und atmosphärischen Einflüssen. Durch die Berücksichtigung von direkter, diffuser und reflektierter Strahlung kann der Energieertrag einer Solaranlage realistisch simuliert werden. Diese Berechnung ist entscheidend für die Planung und Optimierung solarthermischer Anlagen.

\end{document}
